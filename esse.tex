\documentclass[a4paper]{article}
\usepackage[warn]{mathtext}
\usepackage[utf8]{inputenc}
\usepackage[T2A]{fontenc}
\usepackage[english,russian]{babel}
\usepackage{indentfirst}
\usepackage{misccorr}
\usepackage{subcaption}
\captionsetup{compatibility=false}
\usepackage{geometry}
\geometry{verbose,a4paper,tmargin=2cm,bmargin=2cm,lmargin=1.5cm,rmargin=1.5cm}
\usepackage{graphicx}
\usepackage{wrapfig}
\usepackage{amsmath}
\usepackage{fancyhdr}
\usepackage{floatflt}
\usepackage{float}
\usepackage{amssymb}
\usepackage{color}
\usepackage{lscape}
\usepackage{hvfloat}
\usepackage{amsfonts}
\usepackage{euscript}
\usepackage{newunicodechar}
\usepackage{booktabs}
\usepackage{epigraph}
\usepackage{csquotes} 
\usepackage{hyperref}
% \addto\captionsrussian{\def\refname{Список литературы}}
\hypersetup{
    colorlinks=true,      
    urlcolor=black,
    linkcolor= black,
    citecolor = black
}

\begin{document}
\newcommand{\apple}{\char"F8FF}



% \begin{titlepage}
%     \vspace*{4cm}
% 	\centering
%     {\scshape\LARGE Московский физико-технический институт\par}
% 	\vspace{1cm}
% 	{\scshape\Large Дипломная работа\par}
% 	\vspace{1cm}
%     {\huge\bfseries Реализация взаимодействия мобильных агентов в mesh-сети,  обладающей  локальными свойствами. \par}
% 	\vspace{2cm}
% 	\vfill
% \begin{flushright}
% 	{\large Выполнила студентка Б01-907}\par
% 	\vspace{0.3cm}
% 	{\LARGE Юлия Прохорова}
% \end{flushright}
	
% 	\vfill
% Долгопрудный, 2023
% % Bottom of the page
% \end{titlepage}

\pagestyle{fancy} 
\fancyhead[L]{Дипломная работа}
\fancyhead[R]{Юлия Прохорова, Б01-907}
\fancyhead[C]{}
\fancyfoot[C]{ \noindent\rule{\textwidth}{0.4pt} \thepage }

\tableofcontents

\newpage

\epigraph{Mesh: пространство или промежуток между нитями сети.}{Новый английский словарь (М.: Oxford Press, 1932)}

\section{Введение и постановка задачи}
Введем понятие сети как набора объектов (вершин), связанных между собой. Иллюстрацию данного определения можно увидеть на на рис.\ref{p1}. Существуют различные виды сетей: общественные сети, информационные, биологические и т.д. \par
Также сети различают по их топологиям - путям передачи данных в сети:

\begin{figure}[H]
	\begin{center}
	\includegraphics[width=0.4\linewidth]{net.pdf}
	\caption{Пример сети.} 
    \label{p1}
    \end {center}
\end{figure}

\begin{itemize}
    \item Ячеистая (mesh) - каждое устройство подключается к другому устройству через определенный канал;
    \item Топология шины - к кабелю, который выступает в качестве общей разделяемой среды передачи данных, подключены все устройства;
    \item Кольцевая топология - каждое устройство соединяется ровно с двумя соседними, а вместе они образуют кольцо;
    \item Топология звезды - все устройства подключаются к одному сетевому устройству;
    \item Гибридная топология - комбинация различных топологий;
\end{itemize}
В данной работе будут рассматриваться только mesh-сети.
\subsection{Mesh-сети}
Как отображено на pис.\ref{p2}, в mesh-сети все устройства связаны друг с другом через определенный канал.
Mesh-сети бывают: 
\begin{itemize}
    \item Полносвязными (full-mesh);
    \item Неполносвязными (partial-mesh).
\end{itemize}

\begin{figure}[H]
    \text{a)}
	\begin{minipage}[h]{0.45\linewidth}
	\includegraphics[width=1\linewidth]{full.pdf}
	\end{minipage}
	\hfill 
    \text{b)}
	\begin{minipage}[h]{0.43\linewidth}
	\includegraphics[width=1\linewidth]{partial.pdf}
	\end{minipage}
    \caption{\\a)\;Полносвязная сеть. b)\;Неполносвязная сеть.}
    \label{p2}
\end{figure}


\textbf{Полноcвязная топология} \par

В данной топологии все узлы в сети связаны друг с другом. Если в сети имеется $N$ узлов, каждый узел будет иметь $N-1$ соединений.
 Полносвязная сеть обеспечивает избыточность соединений. Поскольку ее реализация будет дороже, чем у неполносвязной, ее обычно резервируют для сетевых магистралей.
 
\textbf{Неполносвязная топология}

В частично связанной сети не обязательно все узлы связаны друг с другомы. Она более практична по сравнению с полносвязной сетью. 
Периферийные сети подключаются с использованием неполносвязной сети и подключаются к полноячеистой магистральной сети.

Топология Mesh основана на децентрализованной схеме организации сети, в отличие от типовых сетей, которые создаются по централизованному принципу.
 Точки доступа, работающие в Mesh-сетях, не только предоставляют услуги абонентского доступа, но и выполняют функции маршрутизаторов/ретрансляторов для других точек доступа той же сети. 

Mesh-сети строятся как совокупность кластеров - зон. Территория, которая покрыта сетями, разделяется на кластерные зоны, число которых теоретически не ограничено. 
Особенностью такой сети является использование специальных протоколов, позволяющих каждой точке доступа создавать таблицы абонентов сети с контролем состояния транспортного канала и поддержкой динамической маршрутизации трафика по оптимальному маршруту между соседними узлами. 
При отказе какого-либо из них происходит автоматическое перенаправление трафика по другому маршруту, что гарантирует не просто доставку трафика адресату, а доставку за минимальное время.

Процедура расширения сети в пределах кластера ограничивается установкой новых точек доступа, интеграция которых в существующую сеть происходит автоматически.

\subsection{Безопасность Mesh-сетей}

Вопросы безопасности Mesh  являются весьма актуальными, особенно для систем городского масштаба, которые объединяют муниципальные, абонентские и корпоративные сети. 
Безопасность сетей обеспечивается в рамках спецификаций стандарта 802.11. Стандарт IEEE 802.11i предусматривает использование в продуктах Wi-Fi таких средств, как поддержка алгоритмов шифрования трафика: TKIP, WRAP и CCMP.
 Этих алгоритмов достаточно для защиты на уровне абонентского трафика, но на уровне корпоративного пользователя используются дополнительные механизмы, включающие более совершенные способы аутентификации при подключении к сети: более крипто-стойкие методы шифрования, динамическую замену ключей шифрования, использование персональных межсетевых экранов, мониторинг защищенности беспроводной сети, технологию виртуальных частных сетей VPN и т.д.
\\

 Основываясь на статье \cite{litlink2} можно сделать следующие выводы:
 \subsection[]{Преимущества ячеистой топологии}

\begin{itemize}
    \item В случае сбоя одного устройства - сеть продолжит функционировать в том же режиме.
    \item Просто локализовать неисправность.
    \item Передача данных более стабильна, так как сбои не выводит всю систему из строя - передача данных реализуется другим путем.
    \item Проблем с трафиком нет, так как для каждого компьютера есть выделенная двухточечная связь.
    \item Эта топология обеспечивает несколько путей для достижения нужного узла и соответственно хорошую устойчивость.
    \item Добавление новых устройств не нарушает передачу данных.
\end{itemize}

\subsection[]{Недостатки ячеистой топологии}
\begin{itemize}
    \item Реализация такой сети дороже по сравнению с иными сетевыми топологиями, т.е. звездой, шиной, двухточечной топологией.
    \item Требуемая мощность оборудования должна быть выше, так как узлы остаются активными все время и распределяют нагрузку между собой.
    \item Некоторые устройства будут избыточными в схеме передачи данных.
\end{itemize}

\section{Сети малого мира}
В статье  М. Е. Дж. Ньюмана, К. Мура и Д. Дж. Уоттса \cite{litlink4} сети малого мира описываюся по аналогии с сетью друзей. 
В сети друзей присутствует “кластеризация”, означающая, что двое ваших друзей с гораздо большей вероятностью также будут друзьями друг друга, чем два человека, выбранных случайным образом из популяции. 
Во-вторых, такая сеть демонстрирует “эффект маленького мира”, а именно, что любые два человека могут установить контакт, пройдя лишь короткую цепочку промежуточных знакомств. 
Американский психолог Милграм провел интересное исследование \cite{litlink5}, в котором определил количество знакомств, необходимое для установления связи между данными людьми. Поскольку Милграм жил в Бостоне, он выбран далекий от Бостона город - Небраска, 
и случайно выбранным людям были розданы конверты, которые нужно было передать в Бостон. Конверты можно было передавать только через своих знакомых и родственников. Милграм получил весьма неожиданный результат: 
в среднем каждый конверт прошел через шесть человек. Так и родилась теория "шести рукопожатий". Т.е. каждый человек связан с любым другим цепью не больше шести личных знакомств. В этом смысле о нашем мире говорят как о малом мире - "small world".

Рассмотрим неориентированную сеть, и определим $l$ как среднее геодезическое (т.е. кратчайшее) расстояние между парами вершин в сети: $$ l = \frac{1}{\frac{1}{2}n(n+1)}\sum_{i \geq j}d_{ij},$$ где $d_{ij}$ - наикоротчайшее расстояние между вершинами $i$ и $j$.
Cуществуют пары вершин, которые не имеют соединительного пути. Обычно таким парам присваивается бесконечное геодезическое расстояние. Значение $l$ также становится бесконечным. Чтобы избежать этой проблемы, в таких сетях обычно определяется среднее геодезическое 
расстояние между всеми парами, которые имеют соединительный путь. Пары, которые попадают в два разных компонента, исключаются из среднего значения. 

Альтернативный способ определения $l$: $$l^{-1} = \frac{1}{\frac{1}{2}n(n+1)}\sum_{i \geq j}d_{ij}^{-1}$$
Эффект малого мира имеет очевидные последствия для динамики процессов, происходящих в сетях.
 Например, если рассматривать распространение информации или чего-либо еще по сети, эффект малого 
 мира подразумевает, что это распространение будет быстрым в большинстве сетей реального мира. 
  Это влияет на количество “прыжков”, которые должен совершить пакет, чтобы добраться с одного компьютера
  на другой в Интернете, количество этапов путешествия для пассажира самолетом или поездом, время,
   необходимое для распространения болезни среди населения, и так далее. 

С другой стороны, эффект малого мира также математически очевиден. Хотя число вершин на расстоянии $r$
от центральной вершины в большинстве случаев растет экспоненциально с увеличением $r$, значение $l$ будет увеличиваться как $logn$. Таким образом, в последние годы термин “эффект малого мира” приобрел 
более точное значение: считается, что сети демонстрируют эффект малого мира, если значение l 
   масштабируется логарифмически или медленнее в зависимости от размера сети при фиксированной средней 
   степени. 
   
 Логарифмическое масштабирование может быть доказано для различных сетевых моделей
     \cite{litlink7,litlink8,litlink9,litlink10,litlink11} а также наблюдался в различных сетях реального мира \cite{litlink12, litlink13, litlink14}. 
    В некоторых сетях средние расстояния между вершинами увеличиваются медленнее, чем log n. Боллоба С.
     и Риордан \cite{litlink15} показали, что сети со степенным распределением степеней (раздел 3.3) имеют значения
      l, которые увеличиваются не быстрее, чем log n/ log log n (см. также \cite{litlink16}), а Коэн и Хэвлин \cite{litlink17} 
      привели аргументы, которые предполагают, что фактическое изменение может быть, даже медленнее, 
      чем это.

Одну из моделей малого мира предложили Ваттс и Строгац \cite{litlink6}. Эта модель представляет собой одномерную регулярную решетку, состоящую из $N$ узлов,
 где каждый узел соединен только со своими k ближайшими соседями и наложены периодические граничные условия, т.е. решетку свернули в кольцо. После чего каждую связь с вероятностью $\phi << C_1$ перебрасывали на другой случайно выбранный узел.
 Правда при такой процедуре есть вероятность появления изолированных узлов.

 \begin{figure}[H]
    \begin{center}
    \text{a)}
	\begin{minipage}[h]{0.4\linewidth}
	\includegraphics[width=1\linewidth]{2соседа.pdf}
	\end{minipage}
	\hfill 
    \text{b)}
	\begin{minipage}[h]{0.4\linewidth}
	\includegraphics[width=1\linewidth]{3neighbours.pdf}
	\end{minipage}
        \caption{Пример малого мира с четырьмя перебросами ($L = 16$) a) - каждый узел соединен со своими ближайшими соседями (к = 2), b) - каждый узел соединен с четырьмя соседями (к = 4)}
    \end{center}
    \label{p3}
\end{figure}

% Пусть $m (r)$ — количество узлов на графике, которые не являются соседними, усредненное по многим реализациям, а $n (r)$ - среднее число “кластеров” вокруг решетки, между которыми разделены эти узлы.
%  В модели континуума и m, и n являются действительными числами. Нам также будет удобно использовать масштабированные переменные: $\mu (r) = \frac{m(r)}{L}$, $\nu (r) = \frac{n(r)}{L}$.

%  В пределе континуума величины $m(r)$ и $n (r)$ удовлетворяют дифференциальным уравнениям следующим образом. Скорость, с которой количество пустых участков на решетке уменьшается с увеличением $r$, равна числу $2n$ растущих ребер кластеров на решетке, умноженному на диапазон $k$ соединений на решетке.
%   Таким образом, $\frac{dm}{dr}=-2kn$ или $\frac{d\mu}{dr} = -2k\nu$.

Формально расстояние до них от любого узла будет бесконечным. Во избежание этого, Ньюман (Newman) и Ваттс предложили связи не перебрасывать, а просто добавлять. Остановимся на этом варианте модели подробнее Среднее расстояние между концами добавленных
связей есть: $\xi = \frac{N}{k\phi N} = \frac{1}{k\phi}$.

Так как существует только один характерный размер системы $X$, то и безразмерное отношение среднего расстояния между узлами графа к числу всех узлов графа $N$ может зависеть только от безразмерной величины: $l = N f( \frac{N}{\xi})$, где $f(x)$ - масштабирующая функция. 
\begin{equation*}
    f(x) = 
    \begin{cases}
       const, x \ll 1 \\
        \frac{\log(x)}{x}, x \gg 1
    \end{cases}
\end{equation*}

Как уже упоминалось выше, существует много способов определения корреляционного радиуса. Предположим, что $\xi \sim \phi^\tau$.
Покажем с помощью ренорм-группового преобразования (для к = 2), что Т = 1. Итак пусть имеем: $l = Nf(Nf^\tau)$.
\subsection{Примеры сетей малого мира}
Свойства маленького мира обнаруживаются во многих явлениях реального мира, включая веб-сайты с меню навигации, пищевые сети, электросети, сети обработки метаболитов, сети нейронов мозга , сети избирателей, телефонный звонок графики, сети аэропортов и социальные сети влияния. Культурные сети, семантические сети и словесные сети совместной встречаемости также оказались сетями небольшого мира.

Сети из связанных белков обладают свойствами небольшого мира, такими как степенное распределение, подчиняющееся распределению степеней. Точно так же транскрипционные сети , в которых узлами являются гены , и они связаны, если один ген оказывает повышающее или понижающее генетическое влияние на другой, обладают свойствами малых мировых сетей.

 \section{Обзор существующих моделей}
\section{Теоретическая часть работы}

\section[]{Реализация}
Даная исследователься работы нацелна создание макета обмена данных в группе мобильных агентов, взаимодействующих в mesh-сети обладающей свойствами “малого мира”.
Возможные технологии для реализации модели:
\begin{itemize}
    \item GNS3
    \item Docker
\end{itemize}
\section{Заключение}

\addcontentsline{toc}{section}{Список литературы}
\begin{thebibliography}{}
    \bibitem{litlink1} M. E. J. Newman,  \href{https://github.com/julproh/Thesis}{The Structure and Function of Complex Networks}
    \bibitem{litlink2} Geeks For Geeks,  \href{https://www.geeksforgeeks.org/advantage-and-disadvantage-of-mesh-topology/}{Advantage and Disadvantage of Mesh Topology}
    \bibitem{litlink3} Осипов И.Е,  \href{http://lib.tssonline.ru/articles2/fix-op/mesh_seti_techn_prilozh_oborud}{Технологии и средства связи \#4, 2006}
    \bibitem{litlink4} M. E. J. Newman, C. Moore, and D. J. Watts,  \href{https://journals.aps.org/prl/abstract/10.1103/PhysRevLett.84.3201}{Mean-Field Solution of the Small-World Network Model, 2000}
    \bibitem{litlink5} S. Milgram, \href{http://snap.stanford.edu/class/cs224w-readings/milgram67smallworld.pdf}{“The small world problem,” Psychol. Today 2, 60–67 (1967)} 
    \bibitem{litlink6} D. J. Watts и S. H. Strogatz, \href{https://www.nature.com/articles/30918}{“Collective dynamics of small-world’ networks,” Nature 393, 440–442 (1998)}
    \bibitem{litlink7} B. Bollobas, The diameter of random graphs, Trans. Amer. Math. Soc., 267 (1981), pp. 41–52.
    \bibitem{litlink8} B. Bollobas, Random Graphs, 2nd ed., Academic Press, New York, 2001.
    \bibitem{litlink9} F. Chung and L. Lu, The average distances in random graphs with given expected degrees, Proc. Natl. Acad. Sci. USA, 99 (2002), pp. 15879–15882.
    \bibitem{litlink10} S. N. Dorogovtsev, J. F. F. Mendes, and A. N. Samukhin, \href{http://arxiv.org/abs/cond-mat/}{ Metric Structure of Random Networks, Preprint 0210085 (2002)}
    \bibitem{litlink11} A. Fronczak, P. Fronczak, and J. A. Holyst, \href{http://arxiv.org/abs/cond-mat/}{Exact Solution for Average Path Length in Random Graphs, Preprint 0212230 (2002)} 
    \bibitem{litlink12} R. Albertand A.-L. Barabasi, Statistical mechanics of complex networks, Rev. Modern
    Phys., 74 (2002), pp. 47–97
    \bibitem{litlink13} M. E. J. Newman, Scientific collaboration networks: II. Shortest paths, weighted networks,
    and centrality, Phys. Rev. E, 64 (2001), art. no. 016132
    \bibitem{litlink14} M. E. J. Newman, The structure of scientific collaboration networks, Proc. Natl. Acad. Sci.
    USA, 98 (2001), pp. 404–409.
    \bibitem{litlink15} B. Bollob as and O. Riordan, The Diameter of a Scale-Free Random Graph, preprint,
    Department of Mathematical Sciences, University of Memphis, 2002
    \bibitem{litlink16} A. Fronczak, P. Fronczak, and J. A. Holyst, \href{http://arxiv.org/abs/cond-mat/}{Exact Solution for Average Path Length
    in Random Graphs, Preprint 0212230 (2002);}
    \bibitem{litlink17} . Cohen and S. Havlin, Scale-free networks are ultrasmall, Phys. Rev. Lett., 90 (2003), art. no. 058701.
\end{thebibliography}
\end{document}

