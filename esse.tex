\documentclass[a4paper]{article}
\usepackage[warn]{mathtext}
\usepackage[utf8]{inputenc}
\usepackage[T2A]{fontenc}
\usepackage[english,russian]{babel}
\usepackage{indentfirst}
\usepackage{misccorr}
\usepackage{subcaption}
\captionsetup{compatibility=false}
\usepackage{geometry}
\geometry{verbose,a4paper,tmargin=2cm,bmargin=2cm,lmargin=1.5cm,rmargin=1.5cm}
\usepackage{graphicx}
\usepackage{wrapfig}
\usepackage{amsmath}
\usepackage{fancyhdr}
\usepackage{floatflt}
\usepackage{float}
\usepackage{amssymb}
\usepackage{color}
\usepackage{lscape}
\usepackage{hvfloat}
\usepackage{amsfonts}
\usepackage{euscript}
\usepackage{newunicodechar}
\usepackage{booktabs}
\usepackage{epigraph}
\usepackage{csquotes} 
\usepackage{hyperref}

\hypersetup{
    colorlinks=true,      
    urlcolor=blue,
    linkcolor= blue
}

\begin{document}
\newcommand{\apple}{\char"F8FF}



\begin{titlepage}
    \vspace*{4cm}
	\centering
    {\scshape\LARGE Московский физико-технический институт\par}
	\vspace{1cm}
	{\scshape\Large Дипломная работа\par}
	\vspace{1cm}
    {\huge\bfseries Реализация взаимодействия мобильных агентов в mesh-сети,  обладающей  локальными свойствами. \par}
	\vspace{2cm}
	\vfill
\begin{flushright}
	{\large Выполнила студентка Б01-907}\par
	\vspace{0.3cm}
	{\LARGE Юлия Прохорова}
\end{flushright}
	
	\vfill
Долгопрудный, 2022
% Bottom of the page
\end{titlepage}

\pagestyle{fancy} 
\fancyhead[L]{Дипломная работа}
\fancyhead[R]{Юля Прохорова, Б01-907}
\fancyhead[C]{}
\fancyfoot[C]{ \noindent\rule{\textwidth}{0.4pt} \thepage }

\tableofcontents

\newpage

\epigraph{Mesh: пространство или промежуток между нитями сети.}{Новый английский словарь (М.: Oxford Press, 1932)}

\section{Введение}
Первостепенно введем понятие \textbf{сети} как набора объектов (вершин), связанных между собой. Существуют различные виды сетей: общественные сети, информационные, биологические и т.д. \par
Также сети различают по их \textbf{топологиям} - путям передачи данных в сети (?вообще есть физическая и логическая топология - мы рассматриваем именно логическую?):
\begin{itemize}
    \item Ячеистая (mesh) - каждое устройство подключается к другому устройству через определенный канал;
    \item Топология шины - к коаксиальному кабелю, который выступает в качестве общей разделяемой среды передачи данных подключены все устройства;
    \item Кольцевая топология - соединяющие устройства ровно с двумя соседними устройствами, образуют кольцо;
    \item Топология звезды - все устройства подключаются к одному сетевому устройству через кабель;
    \item Топология дерева - разновидность топологии звезда;
    \item Гибридная топология - комбинация различных топологий;
\end{itemize}
В данной работе мы будем изучать mesh-сети.
\section{Общая теория}

\subsection{Mesh-сети}
Как уже было упомянуто в mesh-сети все устройства связаны друг с другом через определенный канал.
Mesh-сети бывают: 
\begin{itemize}
    \item Полносвязными (full mesh);
    \item Неполносвязными (partial mesh).
\end{itemize}

\textbf{Полноcвязная топология} \par

В данной топологии все узлы в сети связаны друг с другом. Если в сети имеется N узлов, каждый узел будет иметь n-1ы количество соединений. Полная сетка обеспечивает превосходную степень избыточности, но поскольку ее реализация непомерно дорога, ее обычно резервируют для сетевых магистралей.

Общее количество ссылок, необходимых для ячеистой топологии, равно [n(n-1)]/2.

\textbf{Неполносвязная топология}

Частичная сетка более практична по сравнению с полной сеткой. В частично связанной сетке все узлы не обязательно должны быть связаны друг с другом во время сети. Периферийные сети подключаются с использованием частичной сетки и работают в тандеме с полноячеистой магистральной сетью.

Преимущества ячеистой топологии:

Сбой во время одного устройства не нарушит работу сети.
Проблем с трафиком нет, так как для каждого компьютера есть выделенная двухточечная связь.
Идентификация неисправности проста.
Эта топология обеспечивает несколько путей для достижения цели и тонны избыточности.
Он обеспечивает высокую конфиденциальность и безопасность.
Передача данных более стабильна, потому что сбой не нарушает ее процессов.
Добавление новых устройств не нарушит передачу данных.
Эта топология обладает надежными характеристиками, способными справиться с любой ситуацией.
Сетка не имеет централизованного управления.
Недостатки ячеистой топологии:

Это дорого по сравнению с противоположными сетевыми топологиями, т.е. звездой, шиной, двухточечной топологией.
Установка в сетку крайне затруднена.
Требуемая мощность выше, так как все узлы должны оставаться активными все время и распределять нагрузку.
Сложный процесс.
Стоимость реализации сетки выше других вариантов.
Существует высокий риск избыточных подключений.
Каждый узел требует дополнительных затрат на коммунальные услуги.
Потребности в техническом обслуживании являются сложными с сеткой.

% \begin{figure}[H]
% 	\begin{center}
% 	\begin{minipage}[h]{0.45\linewidth}
% 	\includegraphics[width=1\linewidth]{pic1.pdf}
% 	\caption{Сеть без выделенного доверенного центра.} 
%     \label{p2}
% 	\end{minipage}
% 	\hfill 
% 	\begin{minipage}[h]{0.43\linewidth}
% 	\includegraphics[width=1\linewidth]{pic2.pdf}
% 	\caption{Сеть с выделенным доверенным центром.}
% 	\label{p3}
% 	\end{minipage}
% 	\end{center}
% \end{figure}

\section{Литература}

\begin{thebibliography}{}
    \bibitem{litlink1}  M. E. J. Newman -  \href{https://github.com/julproh/Thesis}{The Structure and Function of Complex Networks}
    \bibitem{litlink2}  Geeks For Geeks -  \href{https://www.geeksforgeeks.org/advantage-and-disadvantage-of-mesh-topology/}{Advantage and Disadvantage of Mesh Topology}
    \bibitem{litlink2}  Осипов И.Е. -  \href{http://lib.tssonline.ru/articles2/fix-op/mesh_seti_techn_prilozh_oborud}{Технологии и средства связи \#4, 2006}
\end{thebibliography}
\end{document}

