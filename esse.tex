\documentclass[a4paper]{article}
\usepackage[warn]{mathtext}
\usepackage[utf8]{inputenc}
\usepackage[T2A]{fontenc}
\usepackage[english,russian]{babel}
\usepackage{indentfirst}
\usepackage{misccorr}
\usepackage{subcaption}
\captionsetup{compatibility=false}
\usepackage{geometry}
\geometry{verbose,a4paper,tmargin=2cm,bmargin=2cm,lmargin=1.5cm,rmargin=1.5cm}
\usepackage{graphicx}
\usepackage{wrapfig}
\usepackage{amsmath}
\usepackage{fancyhdr}
\usepackage{floatflt}
\usepackage{float}
\usepackage{amssymb}
\usepackage{color}
\usepackage{lscape}
\usepackage{hvfloat}
\usepackage{amsfonts}
\usepackage{euscript}
\usepackage{newunicodechar}
\usepackage{booktabs}
\usepackage{epigraph}
\usepackage{csquotes} 
\usepackage{hyperref}

\hypersetup{
    colorlinks=true,      
    urlcolor=blue,
    linkcolor= blue
}

\begin{document}
\newcommand{\apple}{\char"F8FF}



\begin{titlepage}
    \vspace*{4cm}
	\centering
    {\scshape\LARGE Московский физико-технический институт\par}
	\vspace{1cm}
	{\scshape\Large Дипломная работа\par}
	\vspace{1cm}
    {\huge\bfseries Реализация взаимодействия мобильных агентов в mesh-сети,  обладающей  локальными свойствами. \par}
	\vspace{2cm}
	\vfill
\begin{flushright}
	{\large Выполнила студентка Б01-907}\par
	\vspace{0.3cm}
	{\LARGE Юлия Прохорова}
\end{flushright}
	
	\vfill
Долгопрудный, 2022
% Bottom of the page
\end{titlepage}

\pagestyle{fancy} 
\fancyhead[L]{Дипломная работа}
\fancyhead[R]{Юля Прохорова, Б01-907}
\fancyhead[C]{}
\fancyfoot[C]{ \noindent\rule{\textwidth}{0.4pt} \thepage }

\tableofcontents

\newpage

% \epigraph{}{}

\section{Введение}

\section{Общая теория}

\subsection{Mesh-сети}

% \begin{figure}[H]
% 	\begin{center}
% 	\begin{minipage}[h]{0.45\linewidth}
% 	\includegraphics[width=1\linewidth]{pic1.pdf}
% 	\caption{Сеть без выделенного доверенного центра.} 
%     \label{p2}
% 	\end{minipage}
% 	\hfill 
% 	\begin{minipage}[h]{0.43\linewidth}
% 	\includegraphics[width=1\linewidth]{pic2.pdf}
% 	\caption{Сеть с выделенным доверенным центром.}
% 	\label{p3}
% 	\end{minipage}
% 	\end{center}
% \end{figure}

\section{Литература}

\begin{thebibliography}{}
    \bibitem{litlink1}  M. E. J. Newman -  \href{https://lib.mipt.ru/book/n/00013022000cdbe8096da0a688d3a130/Shnaier-B-Prikladnaya-kriptografiya-Protokoly-algoritmy-i-ishodnye-teksty-na-yazyke-S.pdf}{The Structure and Function of Complex Networks}
    \bibitem{litlink2}  Geeks For Geeks -  \href{https://www.geeksforgeeks.org/advantage-and-disadvantage-of-mesh-topology/}{Advantage and Disadvantage of Mesh Topology}
\end{thebibliography}
\end{document}

